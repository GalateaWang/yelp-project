\section{Introduction}
Consumer reviews have revolutionized the way that people choose which businesses
to attend. It is now very common to turn to the web in order to make everyday
decisions, such as where to eat or where to get a haircut.
Yelp\footnote{www.yelp.com} is an opinion and experience sharing system about
businesses of several kinds. In this platform, a user is able to write their
impressions about certain place and rate it with a score from one to five stars.
Therefore, before choosing where to go, someone can investigate what others
think about different places and make a decision with a wider knowledge basis.

To illustrate the importance of such platforms, a survey recently discovered
that 64\% of all consumers scan reviews before choosing a service \cite{survey}.
Businesses' owners, on the other hand, are being forced to direct their
attention to such platforms, since it was found that an extra half star on
average rating increase sales on 19\% \cite{study}.

However, users differ in values and taste. Thus, an opinion might be useful for
someone and not so much for somebody else. Knowing the profile of an individual
is helpful to understand the viewpoint behind a review and decide if its
adequate considering the reader's angle.

The purpose of this work is to identify important reviewers on Yelp for
determined user in order to recommend reviews aligned with the readers'
preferences and style. Different features might be helpful in this process,
including the friendship network of reviewers. In this work, we analyze the
relevance of the friendship network in users profiling as well as of a hidden
friendship network --- defined by users who are similar but are not connected on
the social network. By encountering individuals with similar opinions, validated
by the homophilly on the network, we recommend the reviews of those to the
reader and reduce the burden of manually searching for relevant viewpoints.
